\PassOptionsToPackage{unicode=true}{hyperref} % options for packages loaded elsewhere
\PassOptionsToPackage{hyphens}{url}
%
\documentclass[]{article}
\usepackage{lmodern}
\usepackage[UTF8]{ctex}
\usepackage[justification=centering]{caption}
\usepackage{amssymb,amsmath}
\usepackage{ifxetex,ifluatex}
\usepackage{fixltx2e} % provides \textsubscript
\ifnum 0\ifxetex 1\fi\ifluatex 1\fi=0 % if pdftex
  \usepackage[T1]{fontenc}
  \usepackage[utf8]{inputenc}
  \usepackage{textcomp} % provides euro and other symbols
\else % if luatex or xelatex
  \usepackage{unicode-math}
  \defaultfontfeatures{Ligatures=TeX,Scale=MatchLowercase}
\fi
% use upquote if available, for straight quotes in verbatim environments
\IfFileExists{upquote.sty}{\usepackage{upquote}}{}
% use microtype if available
\IfFileExists{microtype.sty}{%
\usepackage[]{microtype}
\UseMicrotypeSet[protrusion]{basicmath} % disable protrusion for tt fonts
}{}
\IfFileExists{parskip.sty}{%
\usepackage{parskip}
}{% else
\setlength{\parindent}{0pt}
\setlength{\parskip}{6pt plus 2pt minus 1pt}
}
\usepackage{hyperref}
\hypersetup{
            pdfborder={0 0 0},
            breaklinks=true}
\urlstyle{same}  % don't use monospace font for urls
\setlength{\emergencystretch}{3em}  % prevent overfull lines
\providecommand{\tightlist}{%
  \setlength{\itemsep}{0pt}\setlength{\parskip}{0pt}}
\setcounter{secnumdepth}{0}
% Redefines (sub)paragraphs to behave more like sections
\ifx\paragraph\undefined\else
\let\oldparagraph\paragraph
\renewcommand{\paragraph}[1]{\oldparagraph{#1}\mbox{}}
\fi
\ifx\subparagraph\undefined\else
\let\oldsubparagraph\subparagraph
\renewcommand{\subparagraph}[1]{\oldsubparagraph{#1}\mbox{}}
\fi

% set default figure placement to htbp
\makeatletter
\def\fps@figure{htbp}
\makeatother


\date{}

\begin{document}

\hypertarget{header-n0}{%
\section{模糊集理论在投资组合模型中的应用}\label{header-n0}}
赵闯 天津大学管理与经济学部


\textbf{摘要:}投资市场作为一个极其复杂的系统,其环境中大量的不确定性给投资者的决策分析带来了巨大的困难,仅依靠投资市场的历史数据并不能够准确反应风险资产的未来收益,本文从不确定性环境出发研究分析更符合现实交易情况的投资组合模型的构建,具有较高的应用价值。投资组合环境中的不确定性主要包含两种形式:随机性和模糊性。基于随机不确定性的投资组合研究已经发展的相当完善,本文将对基于模糊集理论的投资组合选择研究展开综述。

\textbf{关键词:}投资组合、模糊集理论、隶属度函数、智能算法

\hypertarget{header-n4}{%
\subsection{引言}\label{header-n4}}

一般认为,现代金融理论源自1952年的H.M.Markowitz\textsuperscript{{[}1{]}}提出的投资组合选择理论。``如何在自己能承受的风险下,收益最大化''或``在收益一定下,自己承担的风险最小化''是投资组合中最核心、最重要的两个问题。因此,投资组合被定义为通过运用合理的方法配置各种资产,使其能同时满足投资者的风险与收益的平衡。与此同时,投资者所处的现实环境是高度不确定的,这种不确定性既包括随机不确定性,也包括模糊不确定性。

在现实生活中,随机性主要探讨事物的外在因果,模糊性更倾向于研究事物的内在结构,因此模糊性比随机性的表示更加丰富且深刻\textsuperscript{{[}2{]}}。在过去的的投资组合研究中,基于Markowitz随机均值-方差模型进行拓展的相关学者取得了一系列显著的研究成果\textsuperscript{{[}1{]}}。但其求解时间过长和单纯考虑历史交易数据的局限性导致其在实践中的适用性大大降低。研究者转向探寻其他的模型与方法,其中最具有代表性的是对模糊环境下投资组合模型的探讨,该理论认为现实投资市场中的任意非概率因素均可以利用模糊集理论来进行表示,解决了原有均值-方差模型的固有缺点。因此,在模糊环境下对投资组合模型的探讨受到了大量学者的关注。在理论研究方面,模糊集合理论\textsuperscript{{[}3{]}},可能性理论\textsuperscript{{[}4{]}},对偶犹豫模糊理论\textsuperscript{{[}5{]}}逐渐发展和完善,与投资理论紧密结合,从理论上补充现有的投资组合理论,形成完整的投资组合理论体系。

在本文中,系统的总结了在模糊不确定环境下以及信息不足的情景下投资组合选择问题的相关研究,进一步丰富了现有的投资组合理论体系,同时为日后学者利用模糊集理论进行相关研究指明方向。

\hypertarget{header-n8}{%
\subsection{国内外研究现状}\label{header-n8}}

1952年,Markowitz\textsuperscript{{[}1{]}}定义了投资组合问题的第一个公式,在风险-回报框架中建立了预期均值和方差之间的关系,并使二者达到权衡。自此,大量学者基于Markowitz的理论基础对投资组合收益与风险的度量方法展开研究,并据此搭建了一系列更贴近实际的投资组合研究模型,如Sharpe\textsuperscript{{[}2{]}},Merton\textsuperscript{{[}6{]}},Pang\textsuperscript{{[}7{]}},Perold\textsuperscript{{[}8{]}},Best\textsuperscript{{[}9{]}}。但这一阶段的研究假设均建立在可以由其历史交易信息准确获取收益分布的基础上。然而,由于实际的金融市场中存在大量的非概率因素(包括政治、社会、心理因素等),单凭历史数据并不能够准确反映风险资产的未来收益,因此上述仅使用概率论的方法解决投资组合问题有一定的局限性。随着模糊集理论的兴起,许多学者发现这些非概率因素可以很好的使用模糊集中的相关概念进行表示,越来越多的学者转向了模糊组合模型的研究,如Carlsson\textsuperscript{{[}10{]}},Fangtal\textsuperscript{{[}11{]}},Vercheretal\textsuperscript{{[}12{]}},Barak\textsuperscript{{[}13{]}}等等。其中,Carlsson\textsuperscript{{[}10{]}}在可能性理论框架下推导出可能性均值与可能性方差的表达式,构建了基于可能性理论的投资组合模型。Fangetal\textsuperscript{{[}11{]}}在Carlsson基础上,进一步考虑了交易流动性、成本的影响,同时将证券资产的周转率视作梯形可能性分布,流动性用周转率的可能性均值来表示。在Fangetal基础上,Barak\textsuperscript{{[}13{]}}考虑了收益率组合的三阶中心距,并用其来衡量投资组合的不对称性。然而,学者Huang和Ying\textsuperscript{{[}14{]}}在研究中发现单纯使用模糊变量表示资产收益率会出现一个悖论。为了精准的表示收益率的主管不确定性。Liu\textsuperscript{{[}15{]}}定义了具有自对偶性的可信性测度,并据此提出了不确定理论,作为处理收益率不确定性的另一种替代工具。在该理论中,模糊数的不确定均值、不确定方差和熵的基本概念被首次提出,并给出了不确定半连续法则、模糊模拟、收敛性定理等相关定理和定义。从而使得模糊集的数学基础得到加强,具有了一套公理化体系。自此,基于不确定理论来研究投资组合优化模型成为了投资组合研究的一个新方向。

上文提到的文献绝大多数是单期的投资组合模型,然而,在实践中,大多数投资者均会根据当前状况主动、动态调整自身的投资策略,以应对未来的变化。因此,如何将单周期组合优化问题扩展到多周期组合优化问题是亟待解决的问题,也吸引了许多学者的目光。例如,Celikyurt\textsuperscript{{[}16{]}}
等提出了几个多周期组合优化模型,将无风险资产和风险资产的收益率视为随机变量。Liu
等\textsuperscript{{[}17{]}}使用模糊梯形数来表示风险收益,构建一个模糊的多周期均值-方差投资组合选择模型,并在文中使用三阶矩刻画收益率分布的可能均值。Guo
等\textsuperscript{{[}18{]}}基于可信性理论框架探讨了一种具有 V
型交易成本的模糊多周期投资组合模型,并设计了一种基于模糊仿真的遗传算法来求解多周期均值-方差组合选择模型。Li
等\textsuperscript{{[}19{]}}在不确定性理论框架下建立了模糊多周期均值-方差组合选择模型,并将交易成本和投资者破产纳入模型的考虑范畴。Sadjadi
等\textsuperscript{{[}20{]}}在研究中的主要假设是收益率和借款/贷款利率可以由三角形模糊数刻画。

从现有的文献梳理来看,学者关注的重点是:如何在模糊环境下建立更贴近现实的投资组合模型;如何在必要信息不足的情境下建立投资组合模型;如何有效的求解构建的复杂的、具有模糊背景的投资组合模型。

\hypertarget{header-n12}{%
\subsection{模糊环境下的投资组合模型}\label{header-n12}}

由于现实投资市场中存在许多非概率因素(包括社会、政治、人的心理因素等),仅凭历史交易的数值数据无法准确反映风险资产的未来收益,而这些非概率因素可以使用模糊集理论中的概念进行表示。为此,将投资市场的收益率定义为模糊变量。与此同时,将投资者对风险的主观意愿纳入研究范畴是非常有必要的。1965
年,Zadeh\textsuperscript{{[}21{]}}首次提出模糊集理论。自此,大量学者在模糊情况下对投资组合问题展开进一步研究和探讨。

\hypertarget{header-n14}{%
\subsubsection{模糊集理论}\label{header-n14}}

Zadeh 于 1965
年引入了模糊集合的概念,对集合中的每个元素赋予一个隶属函数,其表示元素隶属于模糊集合的程度,其具体的定义如下:

定义\textsuperscript{{[}21{]}}:若\(\widetilde{A}\)是模糊数,其隶属函数是\(\mu_{\widetilde{A}}\):\(\Re \rightarrow[0,1]\),则有:

(1)\(\widetilde{A}\)具有正规性,即\(\exists x \in R\),使得\(\mu_{\widetilde{A}}=1\);

(2)\(\mu_{\widetilde{A}}\)具有凸性,即当\(\forall x_{1}, x_{2} \in R\),\(\lambda \in [0,1]\),\(\mu_{\tilde{A}}\left[\lambda x_{1}+(1-\lambda) x_{2}\right] \geq \min \left\{\mu_{\tilde{A}}\left(x_{1}\right), \mu_{\widetilde{A}}\left(x_{2}\right)\right\}\);

(3)\(\mu_{\widetilde{A}}\)有界且上半连续,且\(\left\{x \in R \mid \mu_{\widetilde{A}}(x) \leq \varepsilon\right\}\)为闭集;

(4)\(\left\{x \in R \mid \mu_{\widetilde{A}}(x) \geq 0\right\}\)为紧集。

设\(\widetilde{A}\)是模糊数,则\(\widetilde{A}\)的\(\gamma-\)水平截集\([\tilde{A}]^{\gamma}=\left\{x \in R \mid \mu_{\tilde{A}}(x) \geq \gamma\right\}\)可以表示成为\([\widetilde{A}]^{\gamma}=\left[a_{1}(\gamma), a_{2}(\gamma)\right]\)

,其中\(a_{1}(\gamma)=\min \left\{x \in R \mid \mu_{\tilde{A}}(x) \geq \gamma\right\}\),\(a_{2}(\gamma)=\max \left\{x \in R \mid \mu_{\tilde{A}}(x) \geq \gamma\right\}\)。

设\(\tilde{A}=(a, b, \alpha, \beta)\)为模糊数,其中\(L_{\widetilde{A}}\),\(R_{\widetilde{A}}\)是\([0,1]\to[0,1]\)上单调不增的连续函数,且\(L_{\widetilde{A}}(0)=R_{\widetilde{A}}(0)=1\),\(L_{\widetilde{A}}(1)=R_{\widetilde{A}}(1)=0\),则\(\widetilde{A}\)的隶属函数为

\[\mu_{\tilde{A}}=\left\{\begin{array}{ll}
L_{\tilde{A}}\left(\frac{a-x}{\alpha}\right), & a-\alpha \leq x<a \\
1, & a \leq x<b \\
R_{\tilde{A}}\left(\frac{x-b}{\beta}\right), & b \leq x<b+\beta, \\
0, & \text { 其他, }
\end{array}\right.\]

当\(L_{\widetilde{A}}\),\(R_{\widetilde{A}}\)退化为线性函数,此时称模糊数\(\widetilde{A}\)为梯形模糊数,记为\(\tilde{A}=(a, b, \alpha, \beta)\),其隶属函数为:

\[\mu_{\tilde{A}}=\left\{\begin{array}{ll}
1-\frac{a-x}{\alpha}, & a-\alpha \leq x<a \\
1, & a \leq x<b, \\
1-\frac{x-b}{\beta}, & b \leq x<b+\beta, \\
0, & \text { 其他 }
\end{array}\right.\]

当\(L_{\widetilde{A}}\),\(R_{\widetilde{A}}\)退化为线性函数且\(a=b\),此时称模糊数\(\widetilde{A}\)为三角形模糊数,记为\(\tilde{A}=(a, \alpha, \beta)\),其隶属函数为:

\[\mu_{\tilde{A}}=\left\{\begin{array}{ll}
1-\frac{a-x}{\alpha}, & a-\alpha \leq x<a \\
1-\frac{x-b}{\beta}, & b \leq x<b+\beta \\
0, & \text { 其他 }
\end{array}\right.\]

设\(\tilde{A}=(a_{1}, b_{1}, \alpha_{1}, \beta_{1})\)和\(\tilde{B}=(a_{2}, b_{2}, \alpha_{2}, \beta_{2})\)为梯形模糊数,且梯形模糊数\(\widetilde{A}\),\(\widetilde{B}\)的\(\gamma-\)水平截集分别为\([\tilde{A}]^{\gamma}=\left[a_{1}(\gamma), a_{2}(\gamma)\right]\)、\([\tilde{B}]^{\gamma}=\left[b_{1}(\gamma), b_{2}(\gamma)\right]\),则梯形模糊数\(\widetilde{A}\)、\(\widetilde{B}\)的加减法、数乘分别为:

(1)梯形模糊数\(\widetilde{A}\)、\(\widetilde{B}\)的加法:

\[\tilde{A}+\tilde{B}=\left(a_{1}, b_{1}, \alpha_{1}, \beta_{1}\right)+\left(a_{2}, b_{2}, \alpha_{2}, \beta_{2}\right)=\left(a_{1}+a_{2}, b_{1}+b_{2}, \alpha_{1}+\alpha_{2}, \beta_{1}+\beta_{2}\right)\]

且\(\widetilde{A}+\widetilde{B}\)的\(\gamma-\)水平截集为:

\[[\tilde{A}+\tilde{B}]^{\gamma}=\left[a_{1}(\gamma)+b_{1}(\gamma), a_{2}(\gamma)+b_{2}(\gamma)\right], \gamma \in[0,1]\]

(2)模糊数的减法:

\[\tilde{A}-\tilde{B}=\left(a_{1}, b_{1}, \alpha_{1}, \beta_{1}\right)-\left(a_{2}, b_{2}, \alpha_{2}, \beta_{2}\right)=\left(a_{1}-a_{2}, b_{1}-b_{2}, \alpha_{1}-\alpha_{2}, \beta_{1}-\beta_{2}\right)\]

且\(\widetilde{A}-\widetilde{B}\)的\(\gamma-\)水平截集为:

\[[\tilde{A}-\tilde{B}]^{\gamma}=\left[a_{1}(\gamma)-b_{1}(\gamma), a_{2}(\gamma)-b_{2}(\gamma)\right], \gamma \in[0,1]\]

(3)模糊数的数乘:对于任意实数\(\lambda\)和模糊数\(\tilde{A}=(a, b, \alpha, \beta)\),有

\[[\lambda \tilde{A}]^{\gamma}=\lambda[\tilde{A}]^{\gamma}=\lambda\left[a_{1}(\gamma), a_{2}(\gamma)\right]=\left\{\begin{array}{ll}
{\left[\lambda a_{1}(\gamma), \lambda a_{2}(\gamma)\right],} & \lambda \geq 0 \\
{\left[\lambda a_{2}(\gamma), \lambda a_{1}(\gamma)\right],} & \lambda<0
\end{array}\right.\]

\hypertarget{header-n40}{%
\subsubsection{均值-方差模型}\label{header-n40}}

Markowitz
\textsuperscript{{[}1{]}}在风险回报权衡框架中定义了投资组合问题的第一个公式,在文章中建立了预期均值和方差之间的关系,即均值-方差模型(M-V
模型)。Markowitz
认为投资者是理性的,``收益最大化''或``风险最小化'',根据此假设,M-V
模型具体为:

\[\left\{\begin{array}{ll}
\max & X R^{\prime} \\
\text { s.t. } & X V X^{\prime}=\sigma_{0} \\
& X E=1
\end{array}\right.\]

或者

\[\left\{\begin{array}{ll}
\min & X V X^{\prime} \\
\text { s.t. } & X R^{\prime}=\mu \\
& X E=1
\end{array}\right.\]

其中,\(R=\left(E\left(r_{1}\right), E\left(r_{2}\right), \cdots, E\left(r_{n}\right)\right)\),\(E(r_{i})\)为第\(i\)个证券资产的收益率均值,\(V\)是收益率向量的协方差矩阵;\(X=\left(x_{1}, x_{2}, \cdots, x_{n}\right)\),\(x_{i}\)为投资在第\(i\)个证券资产的非负比例;\(E=(1,1,...1)'\);\(\sigma_{0}\)为投资者能承受的最大风险;\(\mu\)为投资者所希望的收益。

有效前沿:是指投资组合模型有效解的集合,它是一条向外凸的曲线。即当投资组合的收益越大,其对应的风险也越大。

\hypertarget{header-n47}{%
\subsubsection{基于模糊理论的投资组合模型}\label{header-n47}}

现有的研究大多将投资组合相关模型和模糊理论进行结合,将不确定因素(如收益)表示为梯形模糊数或者是三角模糊数\textsuperscript{{[}22{]}}。

\textbf{基于梯形模糊数的假设}

对于市场上的股票\(i(i=1,2,3...n)\),设收益率\(r_{i}\)为梯形模糊数:\(r_{i}=(a_{i},b_{i},\alpha_{i},\beta_{i})\),则\(r_{i}\)的\(\gamma-\)截集和期望平均值分别表示为:

\[\left[r_{i}\right]_{\gamma}=\left[a_{i}-(1-\gamma) \alpha_{i}, b_{i}+(1-\gamma) \beta_{i}\right] \equiv\left|a_{i}^{\prime}, a_{i}^{\prime \prime}\right|(\gamma \in[0,1])\]

和

\[E\left(r_{i}\right)=\int_{0}^{1} \gamma\left[a_{i}^{\prime}(\gamma)+a_{i}^{\prime \prime}(\gamma)\right] d \gamma=\frac{a_{i}+b_{i}}{2}+\frac{\beta_{i}-\alpha_{i}}{6}\]

\(r_{i}\)的上、下可能性方差分别为:

\[\operatorname{Var}^{+}\left(r_{i}\right)=\int_{0}^{1} \gamma\left[E\left(r_{i}\right)-a_{i}^{\prime \prime}(\gamma)\right]^{2} d \gamma=\left(\frac{b_{i}-a_{i}}{2}+\frac{\beta_{i}+\alpha_{i}}{6}\right)^{2}+\frac{\beta^{2}}{18}\]

和

\[\operatorname{Var}^{-}\left(r_{i}\right)=\int_{0}^{1} \gamma\left[E\left(r_{i}\right)-a_{i}^{\prime}(\gamma)\right]^{2} d \gamma=\left(\frac{b_{i}-a_{i}}{2}+\frac{\beta_{i}+\alpha_{i}}{6}\right)^{2}+\frac{\alpha^{2}}{18}\]

\(r_{i}\)的上、下可能性协方差分别为:

\[\operatorname{Cov}^{+}\left(r_{i}, r_{j}\right)=2 \int_{0}^{1} \gamma\left[E\left(r_{i}\right)-a_{i}^{\prime \prime}(\gamma)\right] \cdot\left[E\left(r_{j}\right)-a_{j}^{\prime \prime}(\gamma)\right] d \gamma\]

和

\[\operatorname{Cov}^{-}\left(r_{i}, r_{j}\right)=2 \int_{0}^{1} \gamma\left[E\left(r_{i}\right)-a_{i}^{\prime}(\gamma)\right] \cdot\left[E\left(r_{j}\right)-a_{j}^{\prime}(\gamma)\right] d \gamma\]

若使用\(r\)表示投资组合\(x=(x_{1},x_{2},...x_{n})\)的收益率,则该组合的投资总收益的期望表示为:

\[E(x)=E\left(\sum_{i=1}^{n} x_{i} r_{i}\right)=\sum_{i=1}^{n} x_{i} E\left(r_{i}\right)\]

\textbf{基于三角模糊数的假设}

同样也有大量研究将收益率假设为三角形模糊数的形式\textsuperscript{{[}23{]}}:

设三角形模糊数\(A\),其隶属度函数具有形式

\[\mu_{A}(x)=\left\{\begin{array}{cc}
1-\frac{a-x}{\alpha}, & a-\alpha \leq x \leq a \\
1-\frac{x-a}{\beta}, & a \leq x \leq a+\beta, \\
0, & \text { 其他 }
\end{array}\right.\]

相应的,三角形模糊数\(A\)的\(\gamma-\)截集\([A]^{\gamma}=[a-(1-\gamma) \alpha, a+(1-\gamma) \beta], \quad \forall \gamma \in[0,1]\)。

根据定义得到三角模糊数\(A\)的可能性均值和可能性方差分别是

\[E(A)=\int_{0}^{1} \gamma(a+(1-\gamma) \beta+a-(1-\gamma) \alpha) d \gamma=a+\frac{\beta-\alpha}{6}\]

和

\begin{aligned}
\operatorname{Var}(A) &=\int_{0}^{1} \gamma\left[\left(a-(1-\gamma) \alpha-a-\frac{\beta-\alpha}{6}\right)^{2}+\left(a+(1-\gamma) \beta-a-\frac{\beta-\alpha}{6}\right)^{2}\right] d \gamma \\
&=\left[\frac{\alpha+\beta}{6}\right]^{2}+\frac{(\beta+\alpha)^{2}+(\beta-\alpha)^{2}}{72}
\end{aligned}

同样,可以将其带入公式(17)中进行求解收益的期望。

\textbf{数值分析}

将模糊表示的收益率替换原本的历史收益率,对公式(9)或(10)进行求解,求解方法大多为智能算法,如粒子群算法,遗传算法,鱼群算法等,因本文主要讨论模糊集理论在投资组合中的应用,故不展开,可见文献\textsuperscript{{[}24{]}-{[}26{]}}进行了解。

由文献\textsuperscript{{[}24{]}-{[}26{]}}的数值分析结果来看,模糊数是描述具有模糊性和不确定性环境的有利工具,结果相较于传统的概率均值-方差的方法具有显著提升。三角模糊数和梯形模糊数在不同的情景下,有不同的效果,说明在研究过程中,应该根据研究场景确定隶属函数,而不是单纯的选择某一个隶属函数进行求解。

\hypertarget{header-n77}{%
\subsubsection{犹豫模糊理论}\label{header-n77}}

在 1965 年 Zadeh
第一次系统阐述模糊集理论,其用隶属函数表述元素属于某集合的隶属程度。因为在现实世界里存在着高度的模糊不确定性,模糊集理论受到了广泛的关注。但在实际的投资市场中,除了存在模糊不确定性,还存在犹豫不决(即决策者在做决策时会犹豫不决),因此,Torro与
Narukawa\textsuperscript{{[}27{]}}在模糊集理论的基础上对其进行进一步扩展,提出了了犹豫模糊集理论。犹豫模糊集允许各元素的隶属度有多个可能取值,能更加准确的表述决策者的犹豫不决的情形。

令 \(X\) 是一个非空集合,在 \(X\)上的犹豫模糊集(hesitant fuzzy
set,HFS)\(E\)被定义为:

\[E=\left\{\left\langle x, h_{E}(x)\right\rangle \mid x \in X\right\}\]

其中,\(h_{E}(x)\)是\([0,1]\)中的有限元素的集合,表示\(X\)中元素\(x\)隶属于\(E\)可能性的集合。称\(h_{E}(x)\)为犹豫模糊值,记为\(h\),所有犹豫模糊值的集合记为\(H\)。

若每个\(h_{E}(x)\)中只有一个元素存在,则犹豫模糊集就退化为模糊集。

在当前的大部分研究中,很少有学者将犹豫模糊理论和投资组合理论相结合,即并没有将投资者的犹豫不决考虑在模型中。。投资者在投资决策的时候会出现心理和行为偏差,因此,有必要在投资组合模型中考虑投资者的心理、行为偏差,有必要将更贴近现实的模糊理论与投资组合理论相结合,构建更贴近现实的投资组合模型,这将对最终的组合优化结果大有裨益。
本文认为结合犹豫模糊理论的研究方向会成为未来投资组合优化问题的重点研究方向。

\hypertarget{header-n84}{%
\subsection{结论}\label{header-n84}}

模糊集理论作为数学理论的一个分支,为描述各类优化问题中的模糊性提供了新的研究思路和方法。利用模糊理论对投资组合中收益的模糊表示,从而优化其结果,具有重要的实际应用价值和理论知道意义。本文对模糊理论在投资组合选择问题中的常见应用进行了介绍,尤其是隶属度函数选择进行了深入的阐释和探讨。同时,本文在阅读相关文献时发现大量的文献中,存在隶属度函数选择随意,模糊数确定随意的情况,这有悖于模糊性的含义和范畴,后续的研究中应极力避免此情况的发生。最后,本文通过研究模糊集理论在投资组合中的应用,对犹豫模糊集理论的研究前景进行了展望。

\hypertarget{header-n86}{%
\subsection{参考文献}\label{header-n86}}

{[}1{]} Markowitz H M, Todd G P. Mean-variance analysis in portfolio
choice and capital markets{[}M{]}. John Wiley \& Sons, 2000.

{[}2{]} Sharpe W F. Portfolio theory and capital markets{[}M{]}.
McGraw-Hill College, 1970.

{[}3{]} 曾建华, 汪寿阳. 一个基于模糊决策理论的投资组合模型{[}D{]}. ,
2003.

{[}4{]} 史宛蓉. 基于可能性理论的模糊多阶段投资组合选择模型研究{[}D{]}.
南京理工大学, 2019.

{[}5{]} 赵晓冬, 臧誉琪, 王晓倩.
基于对偶犹豫模糊偏好信息的双边稳定匹配决策方法{[}J{]}. 数学的实践与认识,
2018, 48(5): 34-43.

{[}6{]} Merton R C. An analytic derivation of the efficient portfolio
frontier{[}J{]}. Journal of financial and quantitative analysis, 1972:
1851-1872.

{[}7{]} Pang J S. A new and efficient algorithm for a class of portfolio
selection problems{[}J{]}. Operations Research, 1980, 28(3-part-ii):
754-767.

{[}8{]} Perold A F. Large-scale portfolio optimization{[}J{]}.
Management science, 1984, 30(10): 1143-1160.

{[}9{]} Best M J, Hlouskova J. The efficient frontier for bounded
assets{[}J{]}. Mathematical Methods of Operations Research, 2000, 52(2):
195-212.

{[}10{]} Carlsson C, Fullér R, Majlender P. A possibilistic approach to
selecting portfolios with highest utility score{[}J{]}. Fuzzy sets and
systems, 2002, 131(1): 13-21.

{[}11{]} Fang Y, Lai K K, Wang S Y. Portfolio rebalancing model with
transaction costs based on fuzzy decision theory{[}J{]}. European
Journal of Operational Research, 2006, 175(2): 879-893.

{[}12{]} Vercher E, Bermúdez J D, Segura J V. Fuzzy portfolio
optimization under downside risk measures{[}J{]}. Fuzzy sets and
systems, 2007, 158(7): 769-782.

{[}13{]} Barak S, Abessi M, Modarres M. Fuzzy turnover rate chance
constraints portfolio model{[}J{]}. European Journal of Operational
Research, 2013, 228(1): 141-147.

{[}14{]} Huang X, Ying H. Risk index based models for portfolio
adjusting problem with returns subject to experts' evaluations{[}J{]}.
Economic Modelling, 2013, 30: 61-66.

{[}15{]} Liu B. Some research problems in uncertainty theory{[}J{]}.
Journal of Uncertain systems, 2009, 3(1): 3-10.

{[}16{]} Celikyurt U, Özekici S. Multiperiod portfolio optimization
models in stochastic markets using the mean--variance approach{[}J{]}.
European Journal of Operational Research, 2007, 179(1): 186-202.

{[}17{]} Liu Y J, Zhang W G, Xu W J. Fuzzy multi-period portfolio
selection optimization models using multiple criteria{[}J{]}.
Automatica, 2012, 48(12): 3042-3053.

{[}18{]} Guo S, Yu L, Li X, et al. Fuzzy multi-period portfolio
selection with different investment horizons{[}J{]}. European Journal of
Operational Research, 2016, 254(3): 1026-1035.

{[}19{]} Li B, Zhu Y, Sun Y, et al. Multi-period portfolio selection
problem under uncertain environment with bankruptcy constraint{[}J{]}.
Applied Mathematical Modelling, 2018, 56: 539-550.

{[}20{]} Sadjadi S J, Seyedhosseini S M, Hassanlou K. Fuzzy multi period
portfolio selection with different rates for borrowing and
lending{[}J{]}. Applied Soft Computing, 2011, 11(4): 3821-3826.

{[}21{]} Zadeh L A. Fuzzy sets{[}M{]}//Fuzzy sets, fuzzy logic, and
fuzzy systems: selected papers by Lotfi A Zadeh. 1996: 394-432.

{[}22{]} 庄惠丹. 基于投资者主观因素的模糊投资组合模型研究{[}D{]}.
华南理工大学, 2018.

{[}23{]} 高振斌. 基于可能度的模糊证券投资组合优化模型{[}J{]}.
统计与信息论坛, 2015 (2015 年 05): 69-73.

{[}24{]} 宋健. 基于智能算法的模糊投资组合模型及应用研究{[}D{]}.
华南理工大学, 2019.

{[}25{]} 宋健, 邓雪. 基于 PSO-AFSA
混合算法的模糊投资组合问题的研究{[}J{]}. 运筹与管理, 2018, 27(9):
148-155.

{[}26{]} 陈亚波, 李应, 倪丽萍, 等.
改进鱼群算法求解基数约束型投资组合问题{[}J{]}. 计算机工程与设计, 2016,
37(8): 2248-2253.

{[}27{]} Torra V, Narukawa Y. On hesitant fuzzy sets and
decision{[}C{]}//2009 IEEE International Conference on Fuzzy Systems.
IEEE, 2009: 1378-1382.

\end{document}
